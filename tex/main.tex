\documentclass[12pt, reqno]{amsart}
\usepackage{amsmath, amsthm, amscd, amsfonts, amssymb, graphicx, color}
\usepackage[bookmarksnumbered, colorlinks, plainpages]{hyperref}

\textheight 22.5truecm \textwidth 14.5truecm
\setlength{\oddsidemargin}{0.35in}\setlength{\evensidemargin}{0.35in}
\setlength{\topmargin}{-.5cm}
\numberwithin{equation}{section}
\begin{document}
\setcounter{page}{1}



\centerline{}

\centerline{}

\title[Short Title]{Do we even need better page replacement algorithms?}

\author{Mathis Burger}

\begin{abstract}
Developing new page replacement algorithms is quite expensive and time consuming. But do we even need to research for new ones? Or can we
increase the performance of page replacement though other methods and technics more cost efficient for all? This paper takes some simulations
and checks the dependency of three variables to the amount of page faults.
\end{abstract} \maketitle

\section{Introduction and preliminaries}

\noindent Page replacement algorithms are essential when it comes to improving the performance of the computers memory. Therefore, computer scientists developed many
approaches to come even closer to the ideal algorithm as a benchmark. I have asked myself if it is even economically worth it to develop new algorithms. We are able to increase the
size of the memory over the efficiency of PR-algorithms. Is there still a need for new algorithms or will the current algorithms meet our needs in the future of computing?

\section{Simulation conditions}

Through the rise of ubiquitous computing there are many different constellations of hardware configuration.
I took the most common and also took a forefast to possible situations in the future.
\begin{enumerate}
    \item \textbf{Low memory; short reference; low page variety} is a configuration often found in small IOT devices 
    with limited functionality or in smart medical devices with limited functionality.
    \item \textbf{Low memory; long reference; low page variety} as a hardware configuration is often found in cheap IOT devices 
    that are developed to have longer response times, but be cheaper then faster ones.
    \item \textbf{High memory; short reference; low page variety} is mainly used in high response time
    IOT devices or in devices that are designed to execute a certain task very fast.
    \item \textbf{High memory; long reference; low page variety} is often used for standard desktop PCs with small amount of applicatiosn running
    \item \textbf{Low memory; short reference; high page variety} is a possible future scenario for IoT devices with an large amount of data to store and have access to, but with
    limited access.
    \item \textbf{Low memory; long reference; high page variety} is a possible future scenario for IoT devices with an large amount of
    data to process.
    \item \textbf{High memory; short reference; high page variety} is a future scenario for IoT devices or embedded devices that
    may require very high response times.
    \item \textbf{High memory; long reference; high page variety} si also a future scenario for super computers that will have to
    process large amounts of data in very small amount of time. Could also be used for the future of big data.
\end{enumerate} 

\textbf{Note:} Due to very inconsistent charts because of the total amount of data, I had to combine down the data from the raw .xlsx files. All results are broken down
to 50 datapoints or less.


\section{Main results}


\section{Interpretations}

\section{What about prioritized algorithms}
Imagine an algorithm that has priotity built in. Compareable to process management.

\section{Conclusion}

\section{Further information}

\end{document}